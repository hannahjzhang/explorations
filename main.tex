\documentclass{article}
\usepackage[utf8]{inputenc}
\usepackage{amsmath}

% change default 1.875 inch margins to 1 inch
\addtolength{\oddsidemargin}{-.875in}
\addtolength{\evensidemargin}{-.875in}
\addtolength{\textwidth}{1.75in}

\addtolength{\topmargin}{-.875in}
\addtolength{\textheight}{1.75in}
	
\title{\textbf{Test Run}}
\author{Hannah Zhang }
\date{March 14, 2021} 

\begin{document}
\maketitle
\section{Introduction}
This is a test run! \\
Will this `quote' work?! \\
Lot's of random stuff here 

\section{Test}
In March 2006, Congress raised that ceiling an additional \$0.79
trillion to \$8.97 trillion, which is approximately 68\% of GDP. As of
October 4, 2008, the ``Emergency Economic Stabilization Act of
2008'' raised the current debt ceiling to \$11.3 trillion.

\section{Lists and Symbols}
\begin{itemize}  
  \item $\pi r^2$
  \item Subscript: $x_2$
  \item More subscripts: $J_{x-1}$
  \item $\Omega$
  \item $\mu = A e^{Q/RT}$
\end{itemize}

\subsection{Quadratic Formula}
\begin{equation*}
  x = \frac{-b \pm \sqrt{b^2 - 4ac}}{2a}
\end{equation*}
\begin{verbatim}
  This is a cool formula and this has a lot of s p a c e s!
\end{verbatim}

\section{Table}
\begin{tabular}{ l l l }
  Day       & Min Temp & Max Temp \\
  Monday    &    11° C &  22° C \\
  Tuesday   &     9° C &  19° C \\
  Wednesday &    10° C &  21° C \\
\end{tabular}

\end{document}
